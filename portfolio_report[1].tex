\documentclass[14pt]{article}
\usepackage[a4paper, margin=1in]{geometry}
\usepackage{hyperref}
\usepackage{parskip}

\title{Muhammad Mustafa Raza Portfolio Project Report (GH1034254)}
\author{}
\date{}  

\begin{document}

\maketitle

\section*{Introduction}
This report outlines the development and content of my personal portfolio project created as part of the Computer Science Lab module. The purpose of this project is to create a professional portfolio that could be presented when applying for a working student or internship position. The main goal was to learn how to showcase technical and personal skills using real-world tool, such as GitHub, LaTeX, and Jupyter Notebook.

The project required creating a LaTeX-generated CV, a simple Python programming task, a static website built with HTML/CSS, and hosting everything publicly through GitHub Pages and GitHub repositories. This report covers each part of these components, discussing the technologies used, and reflects on what I learned during the process.

\section*{Portfolio Website (HTML \& CSS)}
The portfolio website is created using basic HTML and CSS. It features a light blue background with black and gray text for clear readability and uses a clean font style (Times New Roman). The structure of the website is intentionally simple to make it beginner-friendly and accessible.

The HTML file contain main components of a webpage, such as headings, paragraphs, and lists. These elements are used to divide the content into  sections: About Me, Education, Skills, Interests, and Contact Information. Each section provides basic neccesary information.

The CSS file contains styling instructions. It defines the look for all headings and text, i have opted black font for headings with setting the font size to 16px while the body text is set to 12px in Times New Roman to keep it clear and readable. 

Using separate HTML and CSS files, I learned how structure and style work together in web development. It also helped me understand how to organize content and keep code clean.I realized how putting style rules in a separate file is important to manage and update the webpage.

Hosting this website using GitHub Pages allowed me to publish it online publicly. This process contains steps like  uploading the files to a GitHub repository and enabling the GitHub Pages feature. It was a simple, yet effective way to make the project publicly viewable while practicing deployment methods.

\section*{GitHub Repository}
My GitHub repository includes all files created in this project and is organized for clarity. At the top level, it contains the HTML file for the website, the LaTeX CV files (`main.tex` and `cv.pdf`), and a folder named "python average temprature project"that holds the python code created in jupyter notebook for the Python task.In the last is the CSS file. Ihave also uploaded the latex generated report editable source named by portfolio report.tex.

Each file has been properly named to reflect its purpose. The `index.html` file is the homepage, and style.css contains the styling rules. The `cv.pdf` is the compiled version of my LaTeX CV, and main.tex is the editable source file of the cv. Portfolio report.tex is editable source of portfolio report.

Using GitHub to manage and upload files taught me the importance of version control, clean folder structure, and meaningful filenames. These skills are useful in real-world software development, where many people collaborate on code and rely on GitHub to manage changes.

\section*{LaTeX CV}
One of the most important task of the assignment is the cv CV created using LaTeX. I used Overleaf, a web-based LaTeX editor, to create my CV. The LaTeX a allowed me to create a professional, clean, and consistent layout that is more visually structured than standard Word documents.

The CV contains key sections:: personal details, educational history, technical skills, languages, and interests. I made sure to align the design with the standard expectations of employers in the tech field. The use of bullet points and minimal formatting ensures the CV remains readable and focused.

Creating the CV helped me to get familiar with LaTeX commands, document structure, and formatting. I explored how to modify templates, add bullet lists, bold section headers, and compile the document into a polished PDF. This experience was useful not only for this portfolio but also for future academic reports and formal writing.

\section*{Python Project}
For the programming component, I created a basic Jupyter Notebook using only core Python. The project calculates the average, highest and lowest temperature over a set of five days using a simple list of integers and built-in Python functions such as `sum()`, `len()`, `max()`, and `min()`.

The programming project is compact and straightforward. It demonstrates a clear understanding of Python syntax and fundamental programming concepts. I used a list to store temperature values and applied built-in functions to process and summarize the data. The output is displayed using the `print()` function in the output cells.

The goal was to show skill in basic Python operations and problem solving without relying on libraries. This project helped to reinforce how data can be stored, manipulated, and analyzed with only native Python tools, which is a valuable foundation for future learning.

The Jupyter Notebook file was saved with the `.ipynb` extension and uploaded to the GitHub repository. The code is viewable directly on GitHub, making it easy to verify and understand even without downloading it. I also tested the notebook in Google Colab to confirm that it runs correctly in multiple environments.

\section*{Reflection}
This project gave me a good lesson in building a professional online presence using modern tools. I learned how to write HTML and CSS to design a simple but clear website, how to use GitHub to organize and share my files, and how to use LaTeX for professional document creation.

By creating a Python project and hosting everything online, I developed a clearer understanding of how different technologies can work together to form a complete portfolio. I also practiced using GitHub Pages for live web hosting, which was a new and very practical experience for me. It showed me how developers can present their work publicly and use version control to keep everything organized.

Through trial and error, I also improved my problem-solving ability. For example, I learned how to fix small HTML display errors, upload files to GitHub, and recompile LaTeX documents after making changes. I also learned the value of clear file naming and clean folder structure, which is often underestimated in beginner projects.

If I were to improve this project in the future, I would add more interactive content to the website, such as project showcases with images or embedded videos. I would also like to include a contact form, a downloadable CV button, and more advanced Python projects such as data analysis or visualization using libraries like pandas or matplotlib.

Overall, this assignment gave me practical experience in presenting myself as a computer science student and building a foundation for future career development. I feel more confident using tools like LaTeX, GitHub, and Jupyter, and I now better understand how all these elements work together to create a complete technical portfolio. I have provided link to the website and github repository contaning all files of python, html, css and latex cv.

\section*{Project Links}
\textbf{Live Website:} \url{https://mus3485.github.io/mus3485.portfolio.io/} \\
\textbf{GitHub Repository:} \url{https://github.com/mus3485/mus3485.portfolio.io}

\end{document}
